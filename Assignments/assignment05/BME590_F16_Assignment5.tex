\documentclass[10pt]{report}
\usepackage{epsf}
\usepackage{amsmath}
\usepackage{amssymb}
\usepackage{palatino}
\usepackage[dvips]{graphics}
\usepackage{fancyhdr}
\usepackage{epsfig}
\usepackage{multirow}
\usepackage{multicol}
\usepackage{cancel}
\usepackage{hyperref}
\usepackage{longtable}
\parindent 0in
\parskip 1ex
\oddsidemargin  0in
\evensidemargin 0in
\textheight 8.5in
\textwidth 6.5in
\topmargin -0.25in

\pagestyle{fancy}
\lhead{\bf BME590.06: Medical Software Design}
\rhead{\bf Palmeri \& Kumar (Fall 2017)}
\cfoot{\thepage}


\begin{document}
\section*{Assignment \#5: B-mode Ultrasound Imaging}

{\bf DUE:} Sunday, 2016-11-20 at midnight.

\begin{itemize}

    \item You will be working in your assigned group.
        \footnote{\url{https://gitlab.oit.duke.edu/medical-device-software-design/rw_data}}
        Please create a repository called \verb+ultrasound_netid_netid+ (fill
        in netid of each student member).  Be sure to add Dr.~Palmeri
        (\verb+mlp6+) as a Master access-level member.

    \item Use all good git repository management practices that have been promoted all semester.

    \item Create Issues (that are associated with Milestones) for all
        development tasks on the project, and \textbf{assign a specific group
        member} to each task.  While this is a group project, each group member
        will be graded individually based on their contributions to the
        project, so strive to have even effort distribution, as represented by
        these issues.  \textbf{Be sure to associate commits with specific issues.}

    \item Use all good python coding practices that have been promoted all
        semester, including PEP8 style compliance.

    \item Your group will be developing code to process raw, radiofrequency
        (RF) ultrasound data from a linear array.  The ultrasound physics and
        image formation procedures will be reviewed during lecture.  Your code
        will need to do the following to generate a B-mode ultrasound image:

        \begin{itemize}
            
            \item Data acquisition metadata will be stored in a JSON text file.
                The specific parameters that will be available include:
                \begin{itemize}
                    \item \verb+fs+: sampling frequency (Hz)
                    \item \verb+c+: sound speed (m/s)
                    \item \verb+axial_samples+: number of samples in depth
                    \item \verb+num_beams+: number of lateral beams
                    \item \verb+beam_spacing+: spacing between lateral beams (m)
                \end{itemize}

                Your code should accept an input argument for the JSON
                filename, with a default of \verb+bmode.json+.

            \item The RF data will be serially stored as \verb+int16+ binary
                data ordered as samples from shallow to deep for a single beam,
                followed by subsequent lateral beams from left-to-right in the
                image.   Your code should accept an input argument for the RF
                binary filename, with a default of \verb+rfdat.bin+.

            \item B-mode image formation will involve the steps of envelope
                detection and logarithmic compression discussed in lecture.

            \item Your code should provide an option to either:
                \begin{itemize} 
                    \item Render a B-mode image using \verb+matplotlib+ with
                        axial and lateral dimensions labeled in meters.  This
                        should be done with a \verb+--display+ Boolean input
                        argument that is \verb+False+ by default.

                    \item Save a PNG file of the B-mode image, with the user
                        able to input a desired filename, with a default of
                        \verb+bmode.png+.  This should be done with a
                        \verb+--save+ Boolean input argument that is
                        \verb+True+ by default.

                    \item These options should not be exclusive of one another
                        (i.e., the user can choose to do both).  
                \end{itemize}

        \end{itemize}
    \item Create an annotated tag (\verb+v1.0.0+) of your final version.

\newpage


\item Grading criteria:
\begin{itemize}
    \item Git Repository
        \begin{itemize}
            \item Issues/Milestones [10\%]
            \item Commits are discrete, logical changesets [10\%]
            \item Branching \& Merging [5\%]
        \end{itemize}
    \item Modular coding
        \begin{itemize}
            \item Separate modules for JSON reading, binary reading, envelope
                detection, logarithmic compression, image display, image
                saving. [10\%]
            \item Avoidance of hard-coded variables; robust functional input
                for algorithmic control. [10\%]
        \end{itemize}
    \item Full unit test coverage of all functions, with passing CI
        build\footnote{Dr. Palmeri will enable gitlab runner for your
        repository.} [20\%]
    \item Logging: INFO, DEBUG, ERROR [10\%]
    \item Sphinx documentation for each module/function [10\%]
    \item Handle and raise exceptions [5\%]
    \item Functionality [10\%]
\end{itemize}

\item Test JSON and binary image data are available in:\\
      \url{https://gitlab.oit.duke.edu/medical-device-software-design/bmode_ultrasound}.
\end{itemize}

\end{document}


