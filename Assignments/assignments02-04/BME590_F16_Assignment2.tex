\documentclass[10pt]{report}
\usepackage{epsf}
\usepackage{amsmath}
\usepackage{amssymb}
\usepackage{palatino}
\usepackage[dvips]{graphics}
\usepackage{fancyhdr}
\usepackage{epsfig}
\usepackage{multirow}
\usepackage{multicol}
\usepackage{cancel}
\usepackage{hyperref}
\usepackage{longtable}
\parindent 0in
\parskip 1ex
\oddsidemargin  0in
\evensidemargin 0in
\textheight 8.5in
\textwidth 6.5in
\topmargin -0.25in

\pagestyle{fancy}
\lhead{\bf BME590.06: Medical Software Design}
\rhead{\bf Palmeri \& Kumar (Fall 2017)}
\cfoot{\thepage}


\begin{document}
\section*{Assignment \#2: Heart Rate Monitor}

{\bf DUE:} Sunday, 2016-09-25 at 23:59.

\begin{itemize}

\item Please go back to your Assignment 01 repository (\verb+bme590assignment01+) and add the class grader Net ID \verb+ad306+ as a Master access-level collaborator.

\item Create a new repository--\verb+bme590assignment02+--on GitLab and add Dr.~Palmeri (\verb+mlp6+) and the grader (\verb+ad306+) as Master access-level collaborators.  Add Brenton (\verb+bnk5+) only as needed for help.

\item Building on the skills that we have covered in lecture for the first 2 weeks of class, you will be writing robust code with the following \textbf{functional specifications}:

\begin{itemize}
    \item Estimate instantaneous heart rate.
    \item Estimate 1 \& 5 min average heart rate.
    \item Record 10 min continuous trace log of heart rate.
    \item Generate alarm ``signal'' for brady- or tachycardia.
    \item Be tolerant of noise and \verb+NaN+ values.
    \item Use a combination of ECG and pulse plethysmography data.
\end{itemize}

\item The \textbf{input signal specifications}:
\begin{itemize}
    \item The input signals will be provided as 12-bit unsigned integer data (i.e., a 3 nibble word size) written to a binary file with the ECG and pulse plethysmograph signals being multiplexed every other word in the file.
    \item The first 24 bits (2 words) of the file will represent the sampling frequency of both the ECG and plethysmograph signals in Hz.
\end{itemize}

\item The \textbf{output specifications} to the text terminal:
\begin{itemize}
    \item You code should display the elapsed signal time, updating every 10 seconds.
    \item An instantaneous heart rate estimate should constantly update.
    \item 1 \& 5 min average heart rates should constantly update.
    \item Some indication should be made on the screen when brady- or tachycardia conditions occur.  Your code should be immune to ``false'' alarms.
    \item The 10 min trace of heart rate should be available for inspection after an alarm condition.
\end{itemize}

\item Use good git version control practices when developing this code, including:
\begin{itemize}
    \item Frequent and meaningful commits!  Individual commits should be limited to specific feature implementations, bug fixes, etc.
    \item Project management Milestones \& Issues (with associated git commits)
    \item Unit tests for all algorithmic functions written \underline{before} your actual code.  These should also be associated with related Issue comments.
    \item Feature branches merged into master (ideally after passing unit tests).
    \item Make sure that your project has a \verb+README.txt+ (or \verb+.md+ or \verb+.rst+) file that describes how to run it, and also make sure that you associate a software license with your project (\url{http://choosealicense.com/}).
    \item Create an annotated tag titled ``RC1'' when your assignment is completed and ready to be graded.  (``RC'' stands for ``Release Candidate''; we will ultimately start utilizing Semantic Versioning in our projects (\url{http://semver.org/}).)
\end{itemize}

\item Python code expectations:
\begin{itemize}
    \item Have Sphinx-friendly (\url{https://pythonhosted.org/an_example_pypi_project/sphinx.html}) docstrings for all functions.  These can all be in one file or broken into importable modules.
    \item Your ``main'' file should utilize an \verb+if __name__ == "__main__"+ conditional.
    \item Unit tests should exist in a separate file or directory of test files.  ``py.test'' format tests are preferred, but any Python unit testing package will be accepted if you are more fond of another one.
    \item All functions should have well-defined input-action-output (as the unit tests will demand).
    \item There should be no ``hard-coded'' values in your functions.  Use function input arguments with default values for quantities that should be constant most of the time, or utilize a \verb+**kwargs+ input dictionary.
    \item Follow PEP8 style standards (\url{https://www.python.org/dev/peps/pep-0008/}).  \emph{Note that manually trying to impose compliance will be very challenging; use an IDE or text editor with a PEP8 linter plugin.}
    \item Achieve the functional specifications with passing unit tests.
    \item Gracefully terminate when the input file ends.
\end{itemize}

\item You are purposely \underline{not} being provided with example input signals.  Your code and tests should be able to be developed independently of such signals.

\item Grading criteria:
\begin{itemize}
    \item Effective version control usage [30\%]
    \item Adequate unit test coverage and functional modularity [25\%]
    \item Python style and docstrings [15\%]
    \item Achieves functional specifications [30\%]
    \item \textbf{EXTRA CREDIT:} Instead of displaying your output to a terminal window, use Tkinter (\url{https://wiki.python.org/moin/TkInter}) to create a graphical user interface!  [+15\%]
    \item \emph{No late assignments will be accepted.  Please be sure to name your repository exactly as described above with master access for Dr.~Palmeri and the grader.  Dr.~Palmeri will be cloning all of these repositories at 6 AM Monday morning from France (midnight your time) using an automated script to be able to grade this assignment during his flight, and any access or naming errors will prevent your repository from being cloned!}
\end{itemize}




\end{itemize}

\end{document}