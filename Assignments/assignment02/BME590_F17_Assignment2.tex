\documentclass[10pt]{report}
\usepackage{epsf}
\usepackage{amsmath}
\usepackage{amssymb}
\usepackage{palatino}
\usepackage[dvips]{graphics}
\usepackage{fancyhdr}
\usepackage{epsfig}
\usepackage{multirow}
\usepackage{multicol}
\usepackage{cancel}
\usepackage{hyperref}
\usepackage{longtable}
\parindent 0in
\parskip 1ex
\oddsidemargin  0in
\evensidemargin 0in
\textheight 8.5in
\textwidth 6.5in
\topmargin -0.25in

\pagestyle{fancy}
\lhead{\bf BME590.06: Medical Software Design}
\rhead{\bf Palmeri \& Kumar (Fall 2017)}
\cfoot{\thepage}


\begin{document}
\section*{Assignment \#2: Heart Rate Monitor}

{\bf DUE:} Thursday, 2017-09-21 by the end of class.

\begin{itemize}

\item Create a new repository--\verb+bme590hrm+--on GitHub and add \verb+mlp6+,
\verb+suyashkumar+ and \verb+ad12+ as collaborators.

\item Building on the skills that we have covered in lecture for the first 3
weeks of class, you will be writing robust code with the following
\textbf{functional specifications}:

\begin{itemize}
    \item Read ECG data from a CSV file that will have lines with time (s), lead voltage (mV).  The expected filename should be \verb+ecg_data.csv+, and the first line of this file should be considered a header (comment) line.
    \item In a text output file, your program should save:
    \begin{itemize}
        \item Estimated instantaneous heart rate.
        \item Estimated average heart rate over a user-specified number of minutes.
        \item Indicate when brady- or tachycardia occured in the ECG trace.
    \end{itemize}
\end{itemize}

\item Use good git version control practices when developing this code, including:
\begin{itemize}
    \item Frequent and meaningful commits!  Branches should be used for specific feature implementations, bug fixes, etc.  Do not delete your branches after merging them into master.
    \item Project management Milestones \& Issues (with associated git commits), along with descriptive Labels.
    \item Unit tests for all algorithmic functions written \underline{before} your actual code.  These should also be associated with related Issue comments.
    \item Feature branches merged into master (after passing unit tests with Travis CI).
    \item Make sure that your project has a \verb+README.md+ file that describes how to run it, and also make sure that you associate a software license with your project (\url{http://choosealicense.com/}).  (Bonus - integrate a Travis badge in your README that displays the status of test passage.)
    \item Create an annotated tag titled ``v1.0rc1'' when your assignment is completed and ready to be graded.  (``rc'' stands for ``Release Candidate''; we will ultimately start utilizing Semantic Versioning in our projects (\url{http://semver.org/}).)
\end{itemize}

\item Python code expectations:
\begin{itemize}
    \item Utilize a virtual environment.
    \item Have Sphinx-friendly (\url{https://pythonhosted.org/an_example_pypi_project/sphinx.html}) docstrings for all functions.  These can all be in one file or broken into importable modules.
    \item Your ``main'' file should utilize an \verb+if __name__ == "__main__"+ conditional.
    \item Unit tests should exist in a separate file or directory of test files.  \verb+py.test+ format tests are preferred, but any Python unit testing package will be accepted if you are more fond of another one.
    \item All functions should have well-defined input-action-output (as the unit tests will demand), and be accessible from a module.
    \item There should be no ``hard-coded'' values in your functions.  Use function input arguments with default values for quantities that should be constant most of the time, or utilize a \verb+**kwargs+ input dictionary.
    \item Follow PEP8 style standards (\url{https://www.python.org/dev/peps/pep-0008/}).  \emph{Note that manually trying to impose compliance will be very challenging; use an IDE or text editor with a PEP8 linter plugin.}
    \item Use \verb+try/except+ exception handling as you deem necessary.
    \item Achieve the functional specifications with passing unit tests.
    \item Gracefully terminate when the input file ends.
\end{itemize}

\item You are purposely \underline{not} being provided with example input signals.  Your code and tests should be able to be developed independently of such signals.

\item Grading criteria:
\begin{itemize}
    \item Effective version control usage [3/3]
    \item Adequate unit test coverage and functional modularity [3/3]
    \item Python style and docstrings [3/3]
    \item Achieves functional specifications [3/3]
\end{itemize}

\end{itemize}

\end{document}