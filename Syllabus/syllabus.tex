\documentclass[10pt]{report}
\usepackage{epsf}
\usepackage{amsmath}
\usepackage{amssymb}
\usepackage{palatino}
\usepackage[dvips]{graphics}
\usepackage{fancyhdr}
\usepackage{epsfig}
\usepackage{multirow}
\usepackage{multicol}
\usepackage{cancel}
\usepackage{hyperref}
\usepackage{longtable}
\parindent 0in
\parskip 1ex
\oddsidemargin  0in
\evensidemargin 0in
\textheight 8.5in
\textwidth 6.5in
\topmargin -0.25in

\pagestyle{fancy}
\lhead{\bf BME590.06: Medical Software Design}
\rhead{\bf Palmeri \& Kumar (Fall 2017)}
\cfoot{\thepage}


\begin{document}

\section*{Class Syllabus}

{\bf Instructors:}\\
Dr. Mark Palmeri, M.D., Ph.D.\\
\href{mailto:mark.palmeri@duke.edu}{(mark.palmeri@duke.edu)}\\
Office Hours: \url{https://calendly.com/mark-palmeri}

Suyash Kumar, Uber Software Engineer, Web Guru, Deep Learning Master\\
\href{mailto:suyash.kumar@duke.edu}{(suyash.kumar@duke.edu)}\\
Office Hours: TBD (TBD)

{\bf Teaching Assistants:}\\
Arjun Desai\\
\href{mailto:arjun.desai@duke.edu}{(arjun.desai@duke.edu)}\\
Office Hours: TBD (TBD)

Nisarg Shah\\
\href{mainto:nisarg.shah@duke.edu}{(nisarg.shah@duke.edu)}\\
Office Hours: TBD (TBD)

{\bf Lecture:} Wed/Fri 11:45--13:00, 216 Hudson Hall

\subsection*{Course Overview}
Software plays a critical role in almost all medical devices, spanning device
control, feedback and algorithmic processing.  This course focuses on software
design skills that are ubiquitous in the medical device industry, including
software version control, unit testing, fault tolerance, continuous integration
testing and documentation.  Experience will be gained in Python and JavaScript
(and potentially other languages).

The course will be structured around a project to build an Internet-connected
medical device that measures and processes a biosignal, sends it to a web
server, and makes those data accessible to a web client / mobile application.
This project will be broken into several smaller projects to develop software
design fundamentals.  All project-related work will be done in groups of 3
students.

Prerequisites: Basic familiarity with programming concepts (e.g., variables,
loops, conditional statements)

\section*{Course Objectives}
\begin{multicols}{2}
\begin{itemize}
    \item Define software specifications and constraints
    % \item Agile project management
    % \begin{itemize}
    %    \item Team roles
    %    \item User stories
    %    \item Backlog, To Do, In Progress, Completed, Blocked ``Issues'' (Kanban)
    %    \item Weekly sprints
    %    \item MVPs
    % \end{itemize}
    \item Device programming fundamentals
    \begin{itemize}
        \item Review of data types
        \item Python (v3.6): numpy, scipy, pandas, scikit
        \item Virtual environments \& depedency management (\verb+pip+, \verb+requirements.txt+)
        % \item Simplified Wrapper and Interface Generator (SWIG)
        \item Data management (variables, references, pointers, ASCII/Unicode/binary data)
        \item Regular expressions (regex)
        % \item Compilation, make, cmake
        \item Use of a programming IDE
        \item Debugging (\verb+pudb+)
    \end{itemize}
    \item Backend Software Development in the Cloud
    \begin{itemize}
        % \item HIPPA Compliance in the cloud
        \item Databases (MySQL, MongoDB)
        \item HTTP \& RESTful APIs
        \item Leverage scalable compute infrastructure in the cloud via Remote Procedure Calls (RPCs)
        \item Call web services from Matlab \& Python
        \item Design \& Implementation of a biomedical web service (Python Flask)
        \item Docker and dependency management intro
        \item SSL and Encryption
        \item Internet of Things (IoT) and cloud connected biomedical device design
        % \item Load Balancing and throughput bottlenecks
        \item Sockets and streaming data over networks
    \end{itemize}
    \item Software version control (\verb+git+, GitHub)
    \item Documentation
    \begin{itemize}
        \item Docstrings
        \item Markdown
        \item Sphinx
        \item \url{https://readthedocs.org}
    \end{itemize}
    \item Testing
    \begin{itemize}
        \item Unit testing
        \item Functional / System testing
        \item Continuous integration (Travis CI)
    \end{itemize}
    \item Fault tolerance (raising exceptions)
    \item Logging
    \item Resource profiling (\verb+cProfile+)
\end{itemize}
\end{multicols}

\subsection*{Attendance}
Lecture attendance and participation is important because you will be working in small groups most of the semester.  Participation in these in-class activities will count for 15\% of
your class grade.  It is very understandable that students will have to miss
class for job interviews, personal reasons, illness, etc.  Absences will
be considered \emph{excused} if they are communicated to Dr. Palmeri and Mr. Kumar at least
48 hours in advance (subject to instructor discretion as an excused absence) or, for illness, through submission of
\href{http://www.pratt.duke.edu/undergrad/policies/3531}{Short Term Illness
    Form (STIF)} {\bf before} class.  Unexcused absences will count
against the participation component of your class grade.

\subsection*{Textbooks \& References}
There are no required textbooks for this class.  A variety of online resources will be referenced throughout the semester.  A great resource for an overview of Python:\\
\centerline{\url{https://github.com/jakevdp/WhirlwindTourOfPython}}

\subsection*{Project Details}
Project details will be discussed in lecture throughout the semester.

\subsection*{Grading}
The following grading scheme is subject to change as the semester progresses:

\begin{center}
\begin{tabular}{ll}
Participation                           & 15\% \\
Midterm project deliverables            & 55\% \\
Final project                           & 30\% \\
\end{tabular}
\end{center}

\newpage

\subsection*{Class Schedule}
The course schedule is very likely to change depending on progress throughout
the semester. Specific lecture details, along with deliverable due dates, will
be updated on Sakai and this syllabus
(\url{https://github.com/mlp6/Medical-Software-Design}). New due dates will be
announced in lecture and by Sakai announcements that will be emailed to the
class.  The following is a summary of activities this semester:

\begin{longtable}[c]{|l|l|l|}

    \hline

    \textbf{Date} & \textbf{Lecture} & \textbf{Assignment}\\

    \hline

    Fri Jan 12     &
        Class Introduction, Objectives and Logistics [MLP] &
        A01: Git(Hub), Python Install\\
    \hline
    Wed Jan 17    &
        Git: Repo Setup, Issues, Branching, Pushing/Pulling [MLP] & \\
    Fri Jan 19     &
        Python syntax, virtualenv (\verb+pip+), Unit Testing [MLP] &
        A02: Python Fundamentals \\
    \hline
    Wed Jan 24     &
        Travis CI, Python Exception Handling \& Logging [MLP] & \\
    Fri Jan 26     &
        Python: Dictionaries, Numpy Arrays, Binary Data; &
        A03: Py.test \\
        &
        Docstrings \& Sphinx [MLP] & \\
    \hline
    Wed Jan 31    &
        Classes: Encapsulation \& Composition [MLP] & \\
    Fri Feb 02     &
        Data Types, APIs \& JSON [MLP] &
        A04: Documentation \\
    \hline
    Wed Feb 07    &
        Python: Read/Writing Data (CSV, JSON, HDF5, MATv5) [MLP] & \\
    Fri Feb 09     &
        TBD & A05: Heart Rate Monitor \\
    \hline
    Wed Feb 14     &
        TBD & TBD \\
    Fri Feb 16    &
        TBD & TBD \\
    \hline
    Wed Feb 21     &
        TBD & TBD \\
    Fri Feb 23    &
        TBD & TBD \\
    \hline
    Wed Feb 28      &
        Intro to Web Services \& cloud-connected devices [SK] & \\
    Fri Mar 02    &
        Python Flask, API design, virtual machines (AWS, GCP) [SK] &
        A06: Basic Flask Web Service \\
    \hline
    Wed Mar 07     &
        Microservices, Docker, Flask cont'd [SK] & \\
    Fri Mar 09    &
        Introduction to Databases [SK] &
        A07: Docker Server \\
    \hline
    Wed Mar 14     &
        SPRING BREAK & SPRING BREAK \\
    Fri Mar 16    &
        SPRING BREAK & SPRING BREAK \\
    \hline
    Wed Mar 21    &
        Web, Mobile, Desktop clients (ReactJS) [SK] &
        A08: Cloud Heart Rate Monitor \\
    Fri Mar 23     &
        Web client (ReactJS) cont'd, final project intro [SK] & \\
    \hline
    Wed Mar 28    &
        Software Documentation, Streaming Data, & \\ & Cloud-connected Hardware, SSL [SK] & \\
    Fri Mar 30     &
        IEC 62304 [MLP] & \\
    \hline
    Wed Apr 04     &
        Debugging &
        Working Project Code \\
    Fri Apr 06    &
        TBD, final project work & \\
    \hline
    Wed Apr 11     &
        Project ``Lab'' &
        Refactor Project Code \\
    Fri Apr 13    &
        Final project work & \\
    \hline
    Wed Apr 18    &
        Final project work & \\
    Fri Apr 20    &
        Final project work & \\
    \hline
    Wed Apr 25    &
        Final project due & \\
    \hline


\end{longtable}


\subsection*{Distributed Version Control Software (git)}
Software management is a ubiquitous tool in any engineering project, and this
task becomes increasingly difficult during group development. Version control
software has many benefits and uses in software development, including
preservation of versions during the development process, the ability for
multiple contributors and reviewers on a project, the ability to tag
``releases'' of code, and the ability to branch code into different functional
branches.  We will be using GitHub (\url{https://github.com}) to centrally host
our git repositories.
% Specifically, we will be creating student teams in the Duke BME Design GitHub group (\url{https://github.com/Duke-BME-Design}).  Some guidelines for using your git repositories:

\begin{itemize}
    \item \emph{All} software additions, modifications, bug-fixes, etc.\ need
        to be done in your repository.
    \item The ``Issues'' feature of your repository should be used as a ``to
        do'' list of software-related items, including feature enhancements,
        and bugs that are discovered.
    \item There are several repository management models that we will review in class,
        including branch-development models that need to be used throughout the semester.
    \item Instructors and teaching assistants will only review code that is
    committed to your repository (no emailed code!).
    \item All of the commits associated with your repository are logged with
    your name and a timestamp, and these cannot be modified.  Use descriptive
    commit messages so that your group members, instructors, and teaching
    assistants can figure out what you have done!!  You should not need to email
    group members when you have performed a commit; your commit message(s)
    should speak for themselves.
    \item Code milestones should be properly tagged.
    \item Write software testing routines early in the development process so
        that anyone in your group or an outsider reviewing your code can be
        convinced that it is working as intended.
    \item Modular, modular, modular.
    \item Document!
    \item Make commits small and logical; do them often!
\end{itemize}

We will review working with git repositories in lecture, and feedback
on your software repository will be provided on a regular basis.

\subsection*{Online Slack Channels}
We have online help through the Duke Co-Lab Slack team
(\url{https://colab.duke.edu/}).  We have started two specifics channels for
this class:
\begin{itemize}
  \item \verb+#git+
  \item \verb+#python+
\end{itemize}
Please add yourselves to these channels to get help from your instructors, your
TAs and the Duke community!

\subsection*{Duke Community Standard \& Academic Honor} Engineering is
inherently a collaborative field, and in this class, you are encouraged to work
collaboratively on your projects.  The work that you submit must be the product
of your and your group's effort and understanding.  All resources developed by
another person or company, and used in your project, must be properly
recognized.

All students are expected to adhere to all principles of the
\href{http://www.integrity.duke.edu/standard.html}{Duke Community Standard}.
Violations of the Duke Community Standard will be referred immediately to the
Office of Student Conduct.  Please do not hesitate to talk with your instructors
about any situations involving academic honor, especially if it is ambiguous
what should be done.

\end{document}
